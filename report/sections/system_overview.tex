\section{System overview}

\textit{Describe all the methdodological choices that you performed: e.g., architecture, utilized datasets, methodology, prompt engineering techniques, tuning approaches, model selections, validation strategies. Use separate subparagraphs for each piece of information. Use diagrams to describe architectural choices.
}

This section describes our approach to clarity and evasion classification, including baseline models, transformer architectures, and our multi-task learning strategy.


% Subsection 3.1
\subsection{Dataset and Task Formulation}
\label{subsec:data-repr}

We use the QEvasion dataset~\cite{qevasion}, which contains political question--answer pairs with annotations for two related tasks. The dataset comprises 3,448 train examples and 308 test examples; validation is created as a 10\% split of train (around 345 examples).

\paragraph{Task 1: Clarity Classification.}
Given a question--answer pair, predict whether the answer is a \textit{Clear Reply}, \textit{Clear Non-Reply}, or \textit{Ambivalent} (3-way classification).

\paragraph{Task 2: Evasion Strategy Classification.}
For answers annotated with an evasion label, identify the specific evasion strategy from 9 categories: \textit{Explicit}, \textit{Dodging}, \textit{Implicit}, \textit{General}, \textit{Deflection}, \textit{Declining to answer}, \textit{Claims ignorance}, \textit{Clarification}, and \textit{Partial half-answer}.

\paragraph{Input Representation.}
All models receive the concatenated question and answer text as input. For transformer models, we prepend special tokens to distinguish the two segments: 
\texttt{[CLS] question [SEP] answer [SEP]}.


% Subsection 3.2
\subsection{Baseline Models}
\label{subsec:baselines}

We implement three classical baselines to establish reference performance:

\begin{itemize}
    \item \textbf{Majority Class:} Always predicts the most frequent class in training data.
    \item \textbf{Logistic Regression:} Trained on TF-IDF features (unigrams + bigrams, max 30K features, min-df=2, max-df=0.95) with balanced class weights.
    \item \textbf{Linear SVM:} One-vs-rest linear SVM on the same TF-IDF features with balanced class weights.
\end{itemize}

Both LR and SVM use scikit-learn implementations with default regularization parameters.

\subsection{Transformer-Based Models}
\label{subsec:transformers}

We fine-tune pre-trained transformer encoders using DistilBERT-base-uncased~\cite{sanh2019distilbert} (66M parameters, 6 layers, hidden size 768). We chose DistilBERT for its favorable balance between performance and computational efficiency compared to full BERT models.

\subsubsection{Single-Task Models}

We train separate models for each task:

\paragraph{Clarity Model.}
A standard sequence classification setup with a 3-way output head on top of the \texttt{[CLS]} token representation. Trained with cross-entropy loss:
\[
\mathcal{L}_{\text{clarity}} = -\frac{1}{N}\sum_{i=1}^N \log p_\theta(y_i \mid x_i)
\]

\paragraph{Evasion Model.}
A 9-way classification model trained with Focal Loss~\cite{lin2017focal} to address severe class imbalance in evasion strategies (see Table~\ref{tab:class-dist}). Focal Loss down-weights easy examples and focuses learning on hard cases:
\[
\mathcal{L}_{\text{evasion}} = -\frac{1}{N}\sum_{i=1}^{N} (1 - p_{t,i})^{\gamma} \log(p_{t,i})
\]
where $p_{t,i}$ is the predicted probability for the true class and $\gamma=2.0$ is the focusing parameter. We experimented with $\gamma \in \{0, 1, 2, 3\}$ and found $\gamma=2.0$ optimal on validation data.

\begin{table}[t]
\centering
\small
\begin{tabular}{lcc}
\toprule
\textbf{Evasion Strategy} & \textbf{Train} & \textbf{\%} \\
\midrule
Explicit & 1052 & 30.5\% \\
Dodging & 706 & 20.5\% \\
Implicit & 488 & 14.2\% \\
General & 386 & 11.2\% \\
Deflection & 381 & 11.0\% \\
Declining to answer & 145 & 4.2\% \\
Claims ignorance & 119 & 3.5\% \\
Clarification & 92 & 2.7\% \\
Partial half-answer & 79 & 2.3\% \\


\bottomrule
\end{tabular}
\caption{Class distribution for Task 2 (Evasion) in the training set, showing severe imbalance with 14.4× ratio between most and least frequent classes.}
\label{tab:class-dist}
\end{table}


% Subsection 3.3
\subsubsection{Multi-Task Model}
\label{subsubsec:multitask}

Motivated by the intuition that clarity and evasion strategy are complementary signals, clear non-replies naturally correspond to specific evasion tactics, we develop a multi-task architecture that jointly predicts both labels.

\paragraph{Architecture.}
Our multi-task model shares the DistilBERT encoder across both tasks and employs two task-specific classification heads (Figure~\ref{fig:multitask-arch}). Each head consists of a dropout layer (p=0.1) followed by a linear projection to the respective number of classes.

\begin{figure}[t]
\centering
\resizebox{\columnwidth}{!}{%
    \begin{tikzpicture}[
        node distance=0.8cm,
        box/.style={rectangle, draw, minimum width=3.5cm, minimum height=0.7cm, align=center},
        arrow/.style={->, >=stealth, thick}
    ]
        % Input
        \node[box] (input) {\texttt{[CLS] question [SEP] answer [SEP]}};
        
        % Encoder
        \node[box, below=of input, fill=blue!10] (encoder) {DistilBERT Encoder \\ (shared, 66M params)};
        
        % CLS representation
        \node[box, below=of encoder, fill=green!10] (cls) {\texttt{[CLS]} representation \\ $h \in \mathbb{R}^{768}$};
        
        % Heads
        \node[box, below left=0.8cm and 1.5cm of cls, fill=orange!10] (head1) {Clarity Head \\ (3-way)};
        \node[box, below right=0.8cm and 1.5cm of cls, fill=purple!10] (head2) {Evasion Head \\ (9-way)};
        
        % Outputs
        \node[box, below=of head1] (out1) {$p(\text{clarity} \mid x)$};
        \node[box, below=of head2] (out2) {$p(\text{evasion} \mid x)$};
        
        % Arrows
        \draw[arrow] (input) -- (encoder);
        \draw[arrow] (encoder) -- (cls);
        \draw[arrow] (cls) -- (head1);
        \draw[arrow] (cls) -- (head2);
        \draw[arrow] (head1) -- (out1);
        \draw[arrow] (head2) -- (out2);
    \end{tikzpicture}
}
\caption{Multi-task architecture with shared encoder and task-specific heads.}
\label{fig:multitask-arch}
\end{figure}

\paragraph{Joint Training Objective.}
The model is trained with a weighted sum of task-specific losses:
\[
\mathcal{L}_{\text{total}} = \lambda_{\text{clarity}} \mathcal{L}_{\text{clarity}} + \lambda_{\text{evasion}} \mathcal{L}_{\text{evasion}}
\]
where $\lambda_{\text{clarity}} = \lambda_{\text{evasion}} = 1.0$. We use cross-entropy for clarity and Focal Loss ($\gamma=2.0$) for evasion.

\paragraph{Handling Missing Labels.}
An important challenge is that some examples lack evasion annotations (particularly in the test set, where only clear non-replies are annotated for evasion). We address this by masking the evasion loss:
\[
\mathcal{L}_{\text{evasion}} = \frac{1}{|\mathcal{B}_{\text{valid}}|} \sum_{i \in \mathcal{B}_{\text{valid}}} (1 - p_{t,i})^{\gamma} \log(p_{t,i})
\]
where $\mathcal{B}_{\text{valid}}$ is the subset of examples in the batch with valid evasion labels. This allows the model to learn from all examples for clarity while using only labeled examples for evasion.


% Subsection 3.4
\subsection{Training Configuration}
\label{subsec:training-hparams}

Table~\ref{tab:hyperparams} summarizes our training hyperparameters, selected based on preliminary experiments on the validation set.

\begin{table}[t]
\centering
\small
\begin{tabular}{ll}
\toprule
\textbf{Hyperparameter} & \textbf{Value} \\
\midrule
Optimizer & AdamW \\
Learning rate & $2 \times 10^{-5}$ \\
Batch size & 16 \\
Max epochs & 5 \\
Early stopping patience & 3 epochs \\
Warmup steps & 0 \\
Weight decay & 0.01 \\
Max sequence length & 256 tokens \\
Dropout (task heads) & 0.1 \\
Focal Loss $\gamma$ & 2.0 \\
Loss weights ($\lambda$) & 1.0 (both tasks) \\
Random seed & 42 \\
\bottomrule
\end{tabular}
\caption{Training hyperparameters for transformer models.}
\label{tab:hyperparams}
\end{table}

All models use early stopping based on validation macro F1 score. For the multi-task model, we monitor the average of clarity and evasion macro F1 scores. Inputs are tokenized using the DistilBERT tokenizer with truncation and padding to 256 tokens.


% Subsection 3.5
\subsection{Evaluation Protocol}
\label{subsec:evaluation}

\paragraph{Task 1 (Clarity).}
We report accuracy, macro-averaged F1, weighted F1, precision, and recall on the held-out test set using standard multi-class metrics.

\paragraph{Task 2 (Evasion).}
For training and validation, we use standard single-label evaluation metrics. However, the official test set contains multiple annotator labels per example, which disagrees with the evasion strategy identification. Following the task guidelines, we use \textbf{multi-annotator evaluation}: a prediction is considered correct if it matches any annotator's label for that example. Formally, for predicted label $\hat{y}_i$ and gold label set $G_i$:
\[
\text{Correct}(i) = \mathbb{I}[\hat{y}_i \in G_i]
\]
We compute accuracy, macro F1, and weighted F1 under this relaxed criterion.

\subsection{Implementation Details}

All models are implemented in PyTorch using the Hugging Face Transformers library~\cite{wolf2019transformers}. Experiments are conducted on Google Colab's T4 GPU (16GB GDDR6 VRAM). Training a single transformer model takes approximately 15 minutes. Code and trained models are published.\footnote{\url{https://github.com/minh-de-rien/political-evasion-classifier}}